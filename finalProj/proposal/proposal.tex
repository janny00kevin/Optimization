\documentclass[a4paper, onecolumn, , 11pt]{IEEEtran}

\usepackage[noadjust]{cite}
\usepackage{epsfig}
\usepackage{amsmath}
\usepackage{amssymb}
\usepackage{color}
\usepackage{comment}

\input{Symbol_Shortcut.tex} % Symbols defined by Victor


\newtheorem{theorem}{\bf Theorem}		% by Victor
\newtheorem{corollary}{\bf Corollary}	% by Victor
\newtheorem{definition}{\bf Definition}	% by Victor

\pagenumbering{arabic}
% \pagestyle{empty}

\makeatletter
\def\ifundefined{\@ifundefined}
\makeatother

\setlength{\parskip}{0ex}

\begin{document}

    %\setlength{\baselineskip}{1.18em}
    %\addtolength{\baselineskip}{-0.1ex}
    %\setlength{\baselineskip}{0.95\baselineskip}


    \title{\huge Final Project Proposal}
    \author{312510199 Jing-Hao, Li and 312510197 Zhao-Jie, Luo\\
        Advisor: Professor Carrson C. Fung\\
        \small{Apr. 24, 2024}}
    %\author{Carrson C. Fung$^1$, Man-Wai Kwan$^1$ and Chi-Wah Kok$^2$ \\
    %\small{$^1$Hong Kong App. Sci. \& Tech. Inst., Hong Kong Sci. Park, Shatin,
    %$^2$Dept. of EEE, 
    %Hong Kong Univ. of Sci. \&  Tech., 
    %HONG KONG\\
    %e-mail: \emph{c.fung@ieee.org, mwkwan@ieee.org, eekok@ieee.org}} 
    %\thanks{This work described in this paper has been supported by the Research Grants Council of Hong Kong, China (Project no. CERG HKUST6236/01E)}
    %}

    \markboth{OPTIMIZATION THEORY AND APPLICATIONS}{}

    \maketitle
    % here's how you get a publisher's ID mark with the new
    % IEEEtran.cls.  If you want to use it, don't forget to
    % also uncomment the \pubidadjcol command (which must be
    % executed in the second text column) around line 434 below
    %\pubid{0000--0000/00\$00.00~\copyright~2001 IEEE}

    %\thispagestyle{empty}

    % \begin{comment}
    %     2021 Spring W8 (4/12~4/18)
    %     Coverage: Orthogonality
    % \end{comment}



    \noindent \textbf{Introduction}\\
        We consider a communication system which contains one transmitter and one receiver, and they communicate with each other through a interference channel with 
        Gaussian noise. There are multiple antennas in both transmitter and receiver, implies that it's a MIMO system.\\
        We are interested in finding the channel coefficients so that we can deal with this interference channel by using some methods in the precoder, then we
        can alleviate the interference caused by the channel. \\
        We can send some pilot signals during a time slot, receive these signals which pass through the channel and effected by additive Gaussian noise, 
        then the system model can be written in this form:
        $$\Y = \H\X + \W$$
        Where $\X$ is the transmitted pilot signal, $\W$ is the additive Gaussian noise, $\Y$ is the recieved signal, and $\H$ is the channel coefficients
        that we want to estimate.\\

    \noindent \textbf{Proposed Approach}\\
        We can control $\X$, which is the pilot signal, to have some properties. The mostly used is to let it to be concatenated identity matrix.
        \begin{enumerate}
            \item If the pilot signal $\X$ is a square matrix, which means that it is just a identity matrix, then the estimation $\widehat{\H}$ could be 
                $\Y\X^{-1}$, which is equal to $\Y$.
            \item If $\X$ is a nonsquare concatenated identity matrix, we can obtain $\widehat{\H}= \Y$ multiply the pseudo inverse of $\X$ by using least square method.
            \item To improve performence, \cite{Carrson} suggests that we can use MMSE based primal-dual optimization method with neural network to estimate ${\H}$.
                We want to obtain the minimun mean square error between $\H$ and $\widehat{\H}$. Then we reformulate the problem into epigraph form. We also use 
                parameterize $\widehat{\h}$, which can be a MLP \cite{Eisen2019journal}, so that we can obtain the primal and dual function and thus we can 
                update the primal and dual variables by gradient descent and ascent via policy gradient \cite{Eisen2019journal} until it converges.
            \item However, fully connected neural networks are impractical to both train and implement for large scale networks, as their size grows quickly with 
                network size \cite{Eisen2019conf}, we can use graph neural network \cite{Eisen2020,Naderi2022,Naderi2023,Naderi2024} to be our parameterized 
                channel estimator to see how it waorks under large number of transmitter and receiver antennas.
        \end{enumerate}
        We can compare the performences with these approaches under different SNR and network size.

    % \bibliographystyle{abbrv}
    % \bibliography{mergedBib}
    \vspace{1cm}

    \bibstyle{abbrv}
    \begin{thebibliography}{1}
    \bibitem{Carrson}
        Carrson C. Fung, and Dmytro Ivakhnenkov, "Model-Driven Neural Network Based MIMO Channel Estimator"
    \bibitem{Eisen2019conf}
        M. Eisen and A. Ribeiro, ``Large scale wireless power allocation with graph neural networks,'' \emph{Proc. of the 2019 IEEE 20th Workshop on Signal Processing Advances in Wireless Communications (SPAWC)}, pp. 1-5, 2019.
    \bibitem{Eisen2019journal}
        M. Eisen, C. Zhang, L.F.O. Chamon, D.D. Lee and A. Ribeiro, ``Learning optimal power allocations in wireless systems,'' \emph{IEEE Trans. on Signal Processing}, vol. 67(10), pp. 2775-2790, May 2019.
    \bibitem{Eisen2020}
        M. Eisen and A. Ribeiro, ``Optimal wireless resource allocation with random edge graph neural networks,'' \emph{IEEE Trans. on Signal Processing}, vol. 68, pp. 2977-2991, 2020.
    \bibitem{Naderi2022}    
        N. NaderiAlizadeh, M. Eisen and A. Ribeiro, ``State-Augmented learnable algorithms for resource management in wireless networks,'' \emph{IEEE Trans. on Signal Processing}, vol. 70, pp. 5898-5912, Dec. 2022.
    \bibitem{Naderi2023}    
        N. NaderiAlizadeh, M. Eisen and A. Ribeiro, ``Learning resilient radio resource management policies with graph neural networks,'' \emph{IEEE Trans. on Signal Processing}, volo. 71, pp. 995-1009, Mar. 2023.
    \bibitem{Naderi2024}
        S. Das, N. NaderiAlizadeh and A. Ribeiro, ``State-augmented information routing in communication systems with graph neural networks,''  \emph{IEEE Intl. Conf. on Acoustics, Speech and Signal Processing (ICASSP)}, Seoul, South Korea, Apr. 2024.
        % OpenAI Spinning Up introduction to RL Part 3: Intro to Policy Optimization %: \href{https://spinningup.openai.com/en/latest/spinningup/rl_intro3.html}
    \end{thebibliography}

\end{document}




